\documentclass[11pt,a4paper]{article}

\def\nyear {2025}
\def\nterm {Winter}
\def\nlecturer {}
\def\ncourse {Data Structures}

\makeatletter
\usepackage{amssymb,amsfonts,amsmath,calc,tikz,pgfplots,geometry,mathtools}
\usepackage{color}   % May be necessary if you want to color links
\usepackage[hidelinks]{hyperref}
\usepackage{forest}
\usepackage{commath}
\usepackage{amsthm}
\usepackage{fancyhdr}
\usepackage{bm}
\usepackage{witharrows}
\usepackage{bookmark}
\usepackage{tikz-cd}
\usepackage{bbm}
\usepackage{cleveref}
\pagestyle{fancy}
\usetikzlibrary{positioning}
\usetikzlibrary{matrix}
\usetikzlibrary{automata}
\usetikzlibrary{arrows}
\usetikzlibrary{arrows.meta}
\usetikzlibrary{decorations.markings}
\geometry{margin=1in}
\pgfplotsset{compat=1.18}
\setlength{\headheight}{14.6pt}
\addtolength{\topmargin}{-1.6pt}
\hypersetup{
    colorlinks=false,
    linktoc=section,
    linkcolor=black,
}

\ifx \nauthor\undefined
  \def\nauthor{yehelip}
\else
\fi

\ifx \ncoursehead \undefined
\def\ncoursehead{\ncourse}
\fi

\lhead{\emph{\nouppercase{\leftmark}}}
\ifx \nextra \undefined
  \rhead{
    \ifnum\thepage=1
    \else
      \ncoursehead
    \fi}
\else
  \rhead{
    \ifnum\thepage=1
    \else
      \ncoursehead \ (\nextra)
    \fi}
\fi

\let\@real@maketitle\maketitle
\renewcommand{\maketitle}{\@real@maketitle\begin{center}
\begin{minipage}[c]{0.9\textwidth}\centering\footnotesize
These notes are not endorsed by the lecturers.
I have revised them outside lectures to incorporate supplementary explanations,
clarifications, and material for fun.
While I have strived for accuracy, any errors or misinterpretations 
are most likely mine.
\end{minipage}\end{center}}

%%%%%%%%%%%%%%%%%%%%%%%%%%%%%%%%%%%%%%%%%%%%%%%%%%%%%%%%%%%%%%%%%%%%%%%%%%%%%%%
\theoremstyle{definition}
\newtheorem{definition}{Definition}[section]
\newtheorem{remark}{Remark}[section]
\newtheorem{example}{Example}[section]
\newtheorem{exercise}{Exercise}[section]
\newtheorem{paradox}{Paradox}[section]
\newtheorem*{solution}{Solution}

\theoremstyle{plain}
\newtheorem{theorem}{Theorem}[section]
\newtheorem{proposition}[theorem]{Proposition}
\newtheorem{lemma}[theorem]{Lemma}
\newtheorem{corollary}[theorem]{Corollary}

% \pgfarrowsdeclarecombine{twolatex'}{twolatex'}{latex'}{latex'}{latex'}{latex'}
% \tikzset{->/.style = {decoration={markings,
%                                   mark=at position 1
%                                   with {\arrow[scale=2]{latex'}}},
%                       postaction={decorate}}}
% \tikzset{<-/.style = {decoration={markings,
%                                   mark=at position 0 with {\arrowreversed[scale=2]{latex'}}},
%                       postaction={decorate}}}
% \tikzset{<->/.style = {decoration={markings,
%                                    mark=at position 0 with {\arrowreversed[scale=2]{latex'}},
%                                    mark=at position 1 with {\arrow[scale=2]{latex'}}},
%                        postaction={decorate}}}
% \tikzset{->-/.style = {decoration={markings,
%                                    mark=at position #1 with {\arrow[scale=2]{latex'}}},
%                        postaction={decorate}}}
% \tikzset{-<-/.style = {decoration={markings,
%                                    mark=at position #1 with {\arrowreversed[scale=2]{latex'}}},
%                        postaction={decorate}}}
% \tikzset{->>/.style = {decoration={markings,
%                                   mark=at position 1 with {\arrow[scale=2]{latex'}}},
%                       postaction={decorate}}}
% \tikzset{<<-/.style = {decoration={markings,
%                                   mark=at position 0 with {\arrowreversed[scale=2]{twolatex'}}},
%                       postaction={decorate}}}
% \tikzset{<<->>/.style = {decoration={markings,
%                                    mark=at position 0 with {\arrowreversed[scale=2]{twolatex'}},
%                                    mark=at position 1 with {\arrow[scale=2]{twolatex'}}},
%                        postaction={decorate}}}
% \tikzset{->>-/.style = {decoration={markings,
%                                    mark=at position #1 with {\arrow[scale=2]{twolatex'}}},
%                        postaction={decorate}}}
% \tikzset{-<<-/.style = {decoration={markings,
%                                    mark=at position #1 with {\arrowreversed[scale=2]{twolatex'}}},
%                        postaction={decorate}}}
%
\pgfarrowsdeclare{biggertip}{biggertip}{%
  \setlength{\arrowsize}{1pt}
  \addtolength{\arrowsize}{.1\pgflinewidth}
  \pgfarrowsrightextend{0}
  \pgfarrowsleftextend{-5\arrowsize}
}{%
  \setlength{\arrowsize}{1pt}
  \addtolength{\arrowsize}{.1\pgflinewidth}
  \pgfpathmoveto{\pgfpoint{-5\arrowsize}{4\arrowsize}}
  \pgfpathlineto{\pgfpointorigin}
  \pgfpathlineto{\pgfpoint{-5\arrowsize}{-4\arrowsize}}
  \pgfusepathqstroke
}
\tikzset{
	EdgeStyle/.style = {>=biggertip}
}

\tikzset{circ/.style = {fill, circle, inner sep = 0, minimum size = 3}}
\tikzset{scirc/.style = {fill, circle, inner sep = 0, minimum size = 1.5}}
\tikzset{mstate/.style={circle, draw, black, text=black, minimum width=0.7cm}}

\tikzset{eqpic/.style={baseline={([yshift=-.5ex]current bounding box.center)}}}

\definecolor{mblue}{rgb}{0.2, 0.3, 0.8}
\definecolor{morange}{rgb}{1, 0.5, 0}
\definecolor{mgreen}{rgb}{0, 0.4, 0.2}
\definecolor{mred}{rgb}{0.5, 0, 0}

\DeclareMathOperator{\lcm}{lcm}
\DeclareMathOperator{\idealin}{\triangleleft}
\DeclareMathOperator{\im}{im}
\DeclareMathOperator{\Aut}{Aut}
\DeclareMathOperator{\End}{End}
\DeclareMathOperator{\Inn}{Inn}
\DeclareMathOperator{\Out}{Out}
\DeclareMathOperator{\Mat}{Mat}
\DeclareMathOperator{\std}{std}
\DeclareMathOperator{\Int}{Int}
\DeclareMathOperator{\diam}{diam}
\DeclareMathOperator{\id}{id}
\DeclareMathOperator{\supp}{supp}
\DeclareMathOperator{\Var}{Var} % Variance
\DeclareMathOperator{\Cov}{Cov}
\DeclareMathOperator{\timeop}{time}

\DeclareMathOperator{\Vol}{Vol} % Volume

\DeclareMathOperator{\MOD}{MOD}
\DeclareMathOperator{\Theory}{Theory}

\DeclareMathOperator{\Bin}{Bin}
\DeclareMathOperator{\Geo}{Geo}
\DeclareMathOperator{\Poi}{Poi}
\DeclareMathOperator{\Exp}{Exp}

\newcommand{\dx}{\dif x}
\newcommand{\dt}{\dif t}
\newcommand{\du}{\dif u}
\newcommand{\dv}{\dif v}
\newcommand{\half}{\frac{1}{2}}
\newcommand{\pair}{\del}

\newcommand{\taking}[1]{\xrightarrow{#1}}
\newcommand{\inv}{^{-1}}

% Probability
\newcommand{\Prob}{\mathbf{P}}
\renewcommand{\vec}[1]{\boldsymbol{\mathbf{#1}}}

\newcommand{\ot}{\leftarrow}

\newcommand{\N}{\mathbb{N}}
\newcommand{\Z}{\mathbb{Z}}
\newcommand{\Q}{\mathbb{Q}}
\newcommand{\R}{\mathbb{R}}
\newcommand{\C}{\mathbb{C}}
\newcommand{\F}{\mathbb{F}}
\newcommand{\E}{\mathbb{E}}
\newcommand{\Omicron}{O}

\newcommand{\powerset}{\mathcal{P}}
\renewcommand{\L}{\mathcal{L}}
\newcommand{\ps}{\mathcal{P}}
\newcommand{\M}{\mathcal{M}}

\newcommand{\floor}[1]{\left\lfloor #1 \right\rfloor}
\newcommand{\ceil}[1]{\left\lceil #1 \right\rceil}

\newcommand  \ninfty{-\infty}
\newcommand{\st}{\text{ s.t. }}
\newcommand{\tand}{\quad \text{and} \quad}
\newcommand{\tor}{\quad \text{or} \quad}
\newcommand{\ip}[2]{\langle #1, #2 \rangle}
\renewcommand{\tt}[1]{\textnormal{\textbf{(#1).}}} %tt=theorem title
\newcommand{\bigslant}[2]
{{\raisebox{.2em}{$#1$}\left/\raisebox{-.2em}{$#2$}\right.}}
\title{\textbf{\ncourse}}
\author{Based on lectures by \nlecturer \\\small Notes taken by \nauthor}
\date{\nterm\ \nyear}
\makeatother




\begin{document}
\maketitle

% Insert cool image here

\newpage
\tableofcontents
\newpage

\section{Introduction}

\subsection{Abstract data types}

\begin{example}
	How can we make a data structure for $15,\!000$ students
	that allows insertion and access to students by their id
	in time complexity of $O(1)$.
\end{example}

Data structures appear everywhere.

\begin{definition}[Abstract data type]
    To be added.
\end{definition}

\begin{example}[Stack]
    A stack has multiple basic fuctionalities.
    \begin{itemize}
        \item \texttt{push(S,x)} - add \texttt x to the top of the stack.
        \item \texttt{top(S)} - access top of the stack. Return null
            in the stack is empty.
        \item \texttt{pop(S)} - deletes the element on the top of the
            stack. Does not affect an empty stack.
        \item \texttt{create()} - returns an empty stack.
        \item \texttt{is\_empty(S)} returns an empty stack.
    \end{itemize}
\end{example}

We can implement this abstract data structure using either
an array, or a list. Which is more efficient?

A clear disadventage of an array is that the size of the
stack is limited.

A disadventage of a list is that our memory is limited.

In the real world we use a list of arrays.
We define an array of size $N$ to implement the stack, and when
we reach the maximum capacity of the array, we link
a new array of size $N$.

\begin{exercise}
    What is the absract data type defined by the functions
    \begin{itemize}
        \item \texttt{create(c)} - returns an initial instance.
        \item \texttt{read(c)} - returns an integer.
        \item \texttt{read(create(c))} - returns \texttt 0.
        \item \texttt{increment(c)} - changes \texttt c such that 
            \texttt{read(increment(c) = read(c) + 1}.
    \end{itemize}
\end{exercise}

The surprising answer is a counter!
\begin{itemize}
    \item \texttt{create(c)} - allocates memory to an integer $c$.
    \item \texttt{read(c)} - returns \texttt c.
    \item \texttt{increment(c)} - changes \texttt c such that 
        \texttt{c = c + 1}.
\end{itemize}

But there is a problem since there could be overflow!
How can we solve this?

\begin{example}[Queue]
    A queue has multiple basic fuctionalities.
    \begin{itemize}
        \item \texttt{enqueue(Q,x)} - add $x$ to the end of the queue.
        \item \texttt{head(S)} - access start of the stack. Return null
            in the queue is empty.
        \item \texttt{dequeue(S)} - deletes the first element of the queue,
            and return the first element of the modified queue.
            Does not affect an empty queue.
        \item \texttt{create()} - returns an empty queue.
        \item \texttt{is\_empty($S$)} returns true if the queue is empty.
    \end{itemize}
\end{example}

We can implement a queue using an array, or a linked list.

In \texttt{c++} we can implement an abstract data type in multiple ways, then
by using
% \[
%     typedef ArrayBasedStack Stack;
%     typedef ListBasedStack Stack;
% \]

The quality of the implementation is affected by the time and space complexity
of the functions, and the simplicity of the implementation.

\subsection{Time complexity}

\begin{definition}[Run time]
    The run time of algorithm $A$ on input $x$ is the number of machine
    commands that the algorithm runs on the input.
    We denote it $\timeop_A(x)$.
\end{definition}
\begin{remark}
    We don't need to bother about the different in machines that execute
    the commands, and we also don't bother about the difference in run-time
    of different commands.
\end{remark}

Since we can't possibly write the run time for all inputs, we denote:
\[
    \mathrm t_A(n) = \max\set{\timeop_A(x) \mid |x| = n}.
\]

\begin{example}[For loop]
    Consider the following code:

    we have that
    \[
        n + 1 \le \mathrm t_A(n) \le c_1 \cdot n + c_2.
    \]
    Where $c_1$, $c_2$ are constants that depend on the the implementation
    of the specific programming language.
\end{example}

\begin{example}[check prime]
    Consider the following code:

    We have that
    \begin{align*}
        &\timeop(9888) = c_1 + c_2 \\
        &\timeop(9973) = 9971 \cdot c_1 + c_2 \\
        &\timeop(4) = 9971 \cdot c_1 + c_2 \\
        &\timeop(n) = 10^n \cdot c_1 + c_2
    \end{align*}
\end{example}
\begin{remark}
    Notice that we talk about the \emph{size} of the input and not the 
    \emph{value} of the input.
    Here $n$ is the ?????
\end{remark}

As we said earlier we don't really care about the constants.

We can think about $f(n)$ the function of the real time of the algorithm.

\begin{definition}
    Let $f(n)$, $g(n)$ be two positive functions defined on the natural
    numbers.
    We say that $f(n) \in O(g(n))$ if exist $c > 0$ and $n_0 > 0$ such
    that for all $n > n_0$ we have $f(n) \le c \cdot g(n)$.
\end{definition}

If $f(n) \in O(g(n))$ we usually denote $f(n) = O(g(n))$ and say that
$g(n)$ is an asymptotoic bound of $f(n)$.

\begin{proposition}
    Let $k$ be a constant. Then
    \[
        f(n) = \sum_{i=0}^{k} a_k n^k = O(n^k).
    \]
\end{proposition}
\begin{proof}
    Let $a_{\max} = \max\set{1,a_0,\dots,a_k}$.
    Then we have
    \[
        f(n) \le a_{\max}(1 + n + \dots + n^k) =
        a_{\max} \frac{n^{k+1}}{n-1} \le
        a_{\max} \frac{n^{k+1}}{n/2} = 2 a_{\max} n^k.
    \]
    Thus we can choose $n_0 = 2$ and $c = 2 a_{\max}$ and then
    $f(n) \le c \cdot n^k$ for all $n > n_0$.
\end{proof}

Here are a couple more examples:
\begin{align*}
    &f(n) = 10006 = O(1) \text{ (Constant)} \\
    &f(n) = 4n + 27 =O(n) \text{ (Linear)} \\
    &f(n) = n \log_2(n) + 3n^2 = O(n^2) \text{ (Polynomial)}
\end{align*}

\begin{proposition}
    For all $a,b > 1$ we have that
    \[
        f(n) = \log_a(n) = O(\log_b(n)).
    \]
\end{proposition}

A polynomial is stronger than any logarithm.

\begin{proof}
    For all $\epsilon > 0$ and $a > 1$ we have
    \[ f(n) = \log_a n = \frac{\epsilon \cdot}{a}. \]
\end{proof}

More examples are

\begin{align*}
    &f(n) = a^n = O(b^n) \text{ (Exponential)} \\
    &f(n) = n^k + a^n = O(a^n) \\
    &f(n) = n \cdot log_2 n = O(n \log n)
\end{align*}

We can also show that


\begin{align*}
    &f(n) = 3^n \neq O(2^n) \text{ (Exponential)} \\
    &f(n) = e^n \neq O(n^k) \\
    &f(n) = \log n \neq O(\log \log n)
\end{align*}

\begin{proof}
    Assume by contradiction that exist $c > 0$ and $n_0 \geq 0$ such that
    for all $n > n_0$ we have $3^n \le c \dot 2^n$.
    This implies that $\del{\frac 32}^n \le c$ for all $n > n_0$.
    But this is obviously a contradiction which completes the proof.
\end{proof}

\begin{proof}
    Let $k > 0$.
    We know from analysis that
    \[
        \lim_{x \to \infty} \frac{x^k}{e^x} = 0
    \]
    by L'hopital which imlies that $e^n \neq O(n^k)$ for all $k > 0$
    as wanted.
\end{proof}

\begin{proof}
    Similarly we have that
    \[
        \lim_{n \to \infty} \frac{\log \log n}{\log n} = 0
    \]
    which completes the proof.
\end{proof}

\begin{example}
    Summing elements of an arary of length $n$ is $O(n)$.
\end{example}

\begin{example}[Multiplying matrices]
    When multiplying matrices of order $m \times m$ the complexity
    of the algorithm is $O(m)$.
    But $m$ is not the size of the input.
    Sinc the input is of $2$ matrices it is equal $2m^2$.
    Therefore
    \[
        T(n) = O(m^3) = O\del{\frac{n^{\frac 32}}{2^{\frac 32}}} =
        O(n^{\frac 32})
    \]
\end{example}

\begin{example}[Binary search]
    In a binary search we have that the algorithm executes some commands
    on input of size $n$, and then it executes the same commands on input
    of size
\end{example}

\begin{example}[Weird loops]
    Consider the following code:
    Roughly we have that
    \[
        T(n) \le n (n - 1) = O(n^2).
    \]
    But more precisely we have that
\end{example}

\end{document}
