\documentclass[11pt,a4paper]{article}
\usepackage{amssymb,amsfonts,amsmath,calc,tikz,pgfplots,geometry}
\usepackage{color}   %May be necessary if you want to color links
\usepackage{hyperref}
\usepackage{amsthm}
\usepackage{fancyhdr}
\pagestyle{fancy}
\usetikzlibrary{positioning}
\geometry{margin=1in}
\pgfplotsset{compat=1.18}
\setlength{\headheight}{14.6pt}
\addtolength{\topmargin}{-1.6pt}
\hypersetup{
    colorlinks=false, %set true if you want colored links
    linktoc=all,   %set to all if you want both sections and subsections linked
    linkcolor=black,  %choose some color if you want links to stand out
}
%%%%%%%%%%%%%%%%%%%%%%%%%%%%%%%%%%%%%%%%%%%%%%%%%%%%%%%%%%%%%%%%%%%%%%%%%%%%%%%
\theoremstyle{definition}
\newtheorem{definition}{Definition}[section]
\newtheorem{remark}{Remark}[section]
\newtheorem{example}{Example}[section]
\theoremstyle{plain}
\newtheorem{theorem}{Theorem}[section]
\newtheorem{proposition}[theorem]{Proposition}
\newtheorem{lemma}[theorem]{Lemma}
\newtheorem{corollary}[theorem]{Corollary}

\DeclareMathOperator{\lcm}{lcm}
\DeclareMathOperator{\idealin}{\triangleleft}
\DeclareMathOperator{\im}{im}
\DeclareMathOperator{\Aut}{Aut}
\DeclareMathOperator{\End}{End}
\DeclareMathOperator{\Inn}{Inn}
\DeclareMathOperator{\Out}{Out}
\DeclareMathOperator{\Mat}{Mat}
\DeclareMathOperator{\std}{std}
\DeclareMathOperator{\Int}{Int}
\DeclareMathOperator{\diam}{diam}

\newcommand{\N}{\mathbb{N}}
\newcommand{\Z}{\mathbb{Z}}
\newcommand{\Q}{\mathbb{Q}}
\newcommand{\R}{\mathbb{R}}
\newcommand{\C}{\mathbb{C}}
\newcommand{\F}{\mathbb{F}}
\newcommand{\Omicron}{O}
\newcommand{\st}{\text{ s.t. }}
\newcommand{\tand}{\quad \text{and} \quad}
\newcommand{\tor}{\quad \text{or} \quad}
\newcommand{\ip}[2]{\langle #1, #2 \rangle}
\newcommand{\set}[2]{ \left\{ #1 \mid #2 \right\} }
\newcommand{\abs}[1]{\left\lvert #1\right\rvert}
\newcommand{\norm}[1]{\left\lVert #1\right\rVert}
\renewcommand{\tt}[1]{\textnormal{\textbf{(#1).}}} %tt=theorem title
\newcommand{\bigslant}[2]
{{\raisebox{.2em}{$#1$}\left/\raisebox{-.2em}{$#2$}\right.}}
%%%%%%%%%%%%%%%%%%%%%%%%%%%%%%%%%%%%%%%%%%%%%%%%%%%%%%%%%%%%%%%%%%%%%%%%%%%%%%%
\title{\textbf{Computability Theory}}
\author{}
\date{}
\begin{document}
	\maketitle
	\newpage
  \section{Turing Machines}
  \textbf{The undecidability problem:}\
  Does there exist a ``definite method'' that, 
  when given any possible statement in mathematics, 
  can decide whether that statement is true or false?

  Alan Turing set out with the goal of showing that the answer to this
  problem is no.
  First, in order to prove that there is not algorithm capable of determining
  whether a mathematical statement is false, we need to formally define
  what an algorithm is, and once we have a sensible definition, it could
  be shown that no algorithm under this definition satisfies the
  undecidablity problem.

  In order to mimic the ability to follow algorithms, we find useful
  considering the way we solve problems ourselves.
  We can notice, that when we are faced with a problem, in order to solve
  it we may need to write stuff down, we may need to read stuff, and
  sometimes we may need to perform certain operations with the data
  we gathered. These different operations can be interpreted as states
  of thinking, and the model we are about to define models exactly what
  we described so far.

  \begin{definition}[Turing machine]
    A Turing machine is a $7$-tuple 
    $M = (Q,\Gamma,b,\Sigma,\delta,q_{0},F)$
    such that:
    \begin{itemize}
      \item $Q$ is a finite, nonempty set.
        The elements of $Q$ are called states;
      \item $q_0 \in Q$ is called the initial state;
      \item $F \subseteq Q$ is a set of final states, or accepting states;
      \item $\Gamma$ is a finite, nonempty set such that 
        $Q \cap \Gamma = \emptyset$. The elements of $\Gamma$ are called
        the tape alphabet symbols;
      \item $\Sigma \subsetneq \Gamma$ is called the set of input symbols;
      \item $b \in \Gamma \setminus \Sigma$ is called the blank symbol;
      \item $\delta \colon (Q \setminus F) \times \Gamma \to 
        Q \times \Gamma \times \{L, R, S\}$ is called the transition 
        function.
      \end{itemize}
  \end{definition}

  In order to define a computation on a Turing machine we first have
  to define a configuration (a momentary state) of the machine.

  \begin{definition}[Configuration]
    A configuration (a momentary state) of a Turing machine $M$ is
    a triple $C = (\alpha, q, i)$ such that:
    \begin{itemize}
      \item $q \in Q$ is a state of $M$ called the current state of the
        computation;
      \item $\alpha \in \Gamma^*$ is a finite string of letters from $\Gamma$
        that is called the tape content;
      \item $i \in \N$ is a natural number that is called the head's 
        position.
    \end{itemize}
    Additionally, the initial configuration of $M$ on input $x$ is
    the configuration $(x,q_0,0)$.
  \end{definition}

  \begin{definition}[Finite configuration]
    A finite configuration is any configuration $(\alpha,q,i)$ such that
    $q \in F$.
  \end{definition}

  

  
  



\end{document}
